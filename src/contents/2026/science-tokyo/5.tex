%%%%%%%%%%%%%%%%%%%%%%%%%%%%%%%%%%%%%%%%%%%%%%%%%%
% Metadata
%%%%%%%%%%%%%%%%%%%%%%%%%%%%%%%%%%%%%%%%%%%%%%%%%%
% id: 2026-science-tokyo-5
% title: 2026年 東京科学大学 第5問
% tags: []
% difficulty: C
% source: https://admissions.isct.ac.jp/ja/013/undergraduate/examination-questions

%%%%%%%%%%%%%%%%%%%%%%%%%%%%%%%%%%%%%%%%%%%%%%%%%%
% Preamble
%%%%%%%%%%%%%%%%%%%%%%%%%%%%%%%%%%%%%%%%%%%%%%%%%%
\documentclass[fleqn]{ltjsarticle}

\usepackage{common}
\loadcommonpreamble

% ヘッダー
\lhead{\textbf{2026年 東京科学大学}}

%%%%%%%%%%%%%%%%%%%%%%%%%%%%%%%%%%%%%%%%%%%%%%%%%%
% Document
%%%%%%%%%%%%%%%%%%%%%%%%%%%%%%%%%%%%%%%%%%%%%%%%%%
\begin{document}

\begin{problembox}
    \begin{enumerate} 
        \item [\huge \shikakugo]\quad\raisebox{1ex}{(60点)} \\
        一文字分字数を下げて問題文をここに書く。ここに問題文を書く。ここに問題文を書く。ここに問題文を書く。
        ここに問題文を書く。ここに問題文を書く。ここに問題文を書く。ここに問題文を書く。ここに問題文を書く。
        \begin{enumerate}
            \item [\kakkoichi] ここに小問を書く。ここに小問を書く。ここに小問を書く。
            \item [\kakkoni] ここに小問を書く。ここに小問を書く。ここに小問を書く。
            \item [\kakkosan] ここに小問を書く。ここに小問を書く。ここに小問を書く。
        \end{enumerate}
    \end{enumerate}
\end{problembox}

\begin{multicols}{2}

\noindent
\kaib

\noindent
ここに解答を書く。ここに解答を書く。ここに解答を書く。ここに解答を書く。ここに解答を書く。
ここに解答を書く。ここに解答を書く。ここに解答を書く。ここに解答を書く。ここに解答を書く。
ここに解答を書く。ここに解答を書く。ここに解答を書く。ここに解答を書く。ここに解答を書く。
ここに解答を書く。ここに解答を書く。ここに解答を書く。ここに解答を書く。ここに解答を書く。
ここに解答を書く。ここに解答を書く。ここに解答を書く。ここに解答を書く。ここに解答を書く。
ここに解答を書く。ここに解答を書く。ここに解答を書く。ここに解答を書く。ここに解答を書く。
ここに解答を書く。ここに解答を書く。ここに解答を書く。ここに解答を書く。ここに解答を書く。
ここに解答を書く。ここに解答を書く。ここに解答を書く。ここに解答を書く。ここに解答を書く。

\end{multicols}

\end{document}